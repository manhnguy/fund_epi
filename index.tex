% Options for packages loaded elsewhere
\PassOptionsToPackage{unicode}{hyperref}
\PassOptionsToPackage{hyphens}{url}
\PassOptionsToPackage{dvipsnames,svgnames,x11names}{xcolor}
%
\documentclass[
  letterpaper,
  DIV=11,
  numbers=noendperiod]{scrreprt}

\usepackage{amsmath,amssymb}
\usepackage{lmodern}
\usepackage{iftex}
\ifPDFTeX
  \usepackage[T1]{fontenc}
  \usepackage[utf8]{inputenc}
  \usepackage{textcomp} % provide euro and other symbols
\else % if luatex or xetex
  \usepackage{unicode-math}
  \defaultfontfeatures{Scale=MatchLowercase}
  \defaultfontfeatures[\rmfamily]{Ligatures=TeX,Scale=1}
\fi
% Use upquote if available, for straight quotes in verbatim environments
\IfFileExists{upquote.sty}{\usepackage{upquote}}{}
\IfFileExists{microtype.sty}{% use microtype if available
  \usepackage[]{microtype}
  \UseMicrotypeSet[protrusion]{basicmath} % disable protrusion for tt fonts
}{}
\makeatletter
\@ifundefined{KOMAClassName}{% if non-KOMA class
  \IfFileExists{parskip.sty}{%
    \usepackage{parskip}
  }{% else
    \setlength{\parindent}{0pt}
    \setlength{\parskip}{6pt plus 2pt minus 1pt}}
}{% if KOMA class
  \KOMAoptions{parskip=half}}
\makeatother
\usepackage{xcolor}
\setlength{\emergencystretch}{3em} % prevent overfull lines
\setcounter{secnumdepth}{5}
% Make \paragraph and \subparagraph free-standing
\ifx\paragraph\undefined\else
  \let\oldparagraph\paragraph
  \renewcommand{\paragraph}[1]{\oldparagraph{#1}\mbox{}}
\fi
\ifx\subparagraph\undefined\else
  \let\oldsubparagraph\subparagraph
  \renewcommand{\subparagraph}[1]{\oldsubparagraph{#1}\mbox{}}
\fi

\usepackage{color}
\usepackage{fancyvrb}
\newcommand{\VerbBar}{|}
\newcommand{\VERB}{\Verb[commandchars=\\\{\}]}
\DefineVerbatimEnvironment{Highlighting}{Verbatim}{commandchars=\\\{\}}
% Add ',fontsize=\small' for more characters per line
\usepackage{framed}
\definecolor{shadecolor}{RGB}{241,243,245}
\newenvironment{Shaded}{\begin{snugshade}}{\end{snugshade}}
\newcommand{\AlertTok}[1]{\textcolor[rgb]{0.68,0.00,0.00}{#1}}
\newcommand{\AnnotationTok}[1]{\textcolor[rgb]{0.37,0.37,0.37}{#1}}
\newcommand{\AttributeTok}[1]{\textcolor[rgb]{0.40,0.45,0.13}{#1}}
\newcommand{\BaseNTok}[1]{\textcolor[rgb]{0.68,0.00,0.00}{#1}}
\newcommand{\BuiltInTok}[1]{\textcolor[rgb]{0.00,0.23,0.31}{#1}}
\newcommand{\CharTok}[1]{\textcolor[rgb]{0.13,0.47,0.30}{#1}}
\newcommand{\CommentTok}[1]{\textcolor[rgb]{0.37,0.37,0.37}{#1}}
\newcommand{\CommentVarTok}[1]{\textcolor[rgb]{0.37,0.37,0.37}{\textit{#1}}}
\newcommand{\ConstantTok}[1]{\textcolor[rgb]{0.56,0.35,0.01}{#1}}
\newcommand{\ControlFlowTok}[1]{\textcolor[rgb]{0.00,0.23,0.31}{#1}}
\newcommand{\DataTypeTok}[1]{\textcolor[rgb]{0.68,0.00,0.00}{#1}}
\newcommand{\DecValTok}[1]{\textcolor[rgb]{0.68,0.00,0.00}{#1}}
\newcommand{\DocumentationTok}[1]{\textcolor[rgb]{0.37,0.37,0.37}{\textit{#1}}}
\newcommand{\ErrorTok}[1]{\textcolor[rgb]{0.68,0.00,0.00}{#1}}
\newcommand{\ExtensionTok}[1]{\textcolor[rgb]{0.00,0.23,0.31}{#1}}
\newcommand{\FloatTok}[1]{\textcolor[rgb]{0.68,0.00,0.00}{#1}}
\newcommand{\FunctionTok}[1]{\textcolor[rgb]{0.28,0.35,0.67}{#1}}
\newcommand{\ImportTok}[1]{\textcolor[rgb]{0.00,0.46,0.62}{#1}}
\newcommand{\InformationTok}[1]{\textcolor[rgb]{0.37,0.37,0.37}{#1}}
\newcommand{\KeywordTok}[1]{\textcolor[rgb]{0.00,0.23,0.31}{#1}}
\newcommand{\NormalTok}[1]{\textcolor[rgb]{0.00,0.23,0.31}{#1}}
\newcommand{\OperatorTok}[1]{\textcolor[rgb]{0.37,0.37,0.37}{#1}}
\newcommand{\OtherTok}[1]{\textcolor[rgb]{0.00,0.23,0.31}{#1}}
\newcommand{\PreprocessorTok}[1]{\textcolor[rgb]{0.68,0.00,0.00}{#1}}
\newcommand{\RegionMarkerTok}[1]{\textcolor[rgb]{0.00,0.23,0.31}{#1}}
\newcommand{\SpecialCharTok}[1]{\textcolor[rgb]{0.37,0.37,0.37}{#1}}
\newcommand{\SpecialStringTok}[1]{\textcolor[rgb]{0.13,0.47,0.30}{#1}}
\newcommand{\StringTok}[1]{\textcolor[rgb]{0.13,0.47,0.30}{#1}}
\newcommand{\VariableTok}[1]{\textcolor[rgb]{0.07,0.07,0.07}{#1}}
\newcommand{\VerbatimStringTok}[1]{\textcolor[rgb]{0.13,0.47,0.30}{#1}}
\newcommand{\WarningTok}[1]{\textcolor[rgb]{0.37,0.37,0.37}{\textit{#1}}}

\providecommand{\tightlist}{%
  \setlength{\itemsep}{0pt}\setlength{\parskip}{0pt}}\usepackage{longtable,booktabs,array}
\usepackage{calc} % for calculating minipage widths
% Correct order of tables after \paragraph or \subparagraph
\usepackage{etoolbox}
\makeatletter
\patchcmd\longtable{\par}{\if@noskipsec\mbox{}\fi\par}{}{}
\makeatother
% Allow footnotes in longtable head/foot
\IfFileExists{footnotehyper.sty}{\usepackage{footnotehyper}}{\usepackage{footnote}}
\makesavenoteenv{longtable}
\usepackage{graphicx}
\makeatletter
\def\maxwidth{\ifdim\Gin@nat@width>\linewidth\linewidth\else\Gin@nat@width\fi}
\def\maxheight{\ifdim\Gin@nat@height>\textheight\textheight\else\Gin@nat@height\fi}
\makeatother
% Scale images if necessary, so that they will not overflow the page
% margins by default, and it is still possible to overwrite the defaults
% using explicit options in \includegraphics[width, height, ...]{}
\setkeys{Gin}{width=\maxwidth,height=\maxheight,keepaspectratio}
% Set default figure placement to htbp
\makeatletter
\def\fps@figure{htbp}
\makeatother
\newlength{\cslhangindent}
\setlength{\cslhangindent}{1.5em}
\newlength{\csllabelwidth}
\setlength{\csllabelwidth}{3em}
\newlength{\cslentryspacingunit} % times entry-spacing
\setlength{\cslentryspacingunit}{\parskip}
\newenvironment{CSLReferences}[2] % #1 hanging-ident, #2 entry spacing
 {% don't indent paragraphs
  \setlength{\parindent}{0pt}
  % turn on hanging indent if param 1 is 1
  \ifodd #1
  \let\oldpar\par
  \def\par{\hangindent=\cslhangindent\oldpar}
  \fi
  % set entry spacing
  \setlength{\parskip}{#2\cslentryspacingunit}
 }%
 {}
\usepackage{calc}
\newcommand{\CSLBlock}[1]{#1\hfill\break}
\newcommand{\CSLLeftMargin}[1]{\parbox[t]{\csllabelwidth}{#1}}
\newcommand{\CSLRightInline}[1]{\parbox[t]{\linewidth - \csllabelwidth}{#1}\break}
\newcommand{\CSLIndent}[1]{\hspace{\cslhangindent}#1}

\KOMAoption{captions}{tableheading}
\makeatletter
\makeatother
\makeatletter
\@ifpackageloaded{bookmark}{}{\usepackage{bookmark}}
\makeatother
\makeatletter
\@ifpackageloaded{caption}{}{\usepackage{caption}}
\AtBeginDocument{%
\ifdefined\contentsname
  \renewcommand*\contentsname{Table of contents}
\else
  \newcommand\contentsname{Table of contents}
\fi
\ifdefined\listfigurename
  \renewcommand*\listfigurename{List of Figures}
\else
  \newcommand\listfigurename{List of Figures}
\fi
\ifdefined\listtablename
  \renewcommand*\listtablename{List of Tables}
\else
  \newcommand\listtablename{List of Tables}
\fi
\ifdefined\figurename
  \renewcommand*\figurename{Figure}
\else
  \newcommand\figurename{Figure}
\fi
\ifdefined\tablename
  \renewcommand*\tablename{Table}
\else
  \newcommand\tablename{Table}
\fi
}
\@ifpackageloaded{float}{}{\usepackage{float}}
\floatstyle{ruled}
\@ifundefined{c@chapter}{\newfloat{codelisting}{h}{lop}}{\newfloat{codelisting}{h}{lop}[chapter]}
\floatname{codelisting}{Listing}
\newcommand*\listoflistings{\listof{codelisting}{List of Listings}}
\makeatother
\makeatletter
\@ifpackageloaded{caption}{}{\usepackage{caption}}
\@ifpackageloaded{subcaption}{}{\usepackage{subcaption}}
\makeatother
\makeatletter
\@ifpackageloaded{tcolorbox}{}{\usepackage[many]{tcolorbox}}
\makeatother
\makeatletter
\@ifundefined{shadecolor}{\definecolor{shadecolor}{rgb}{.97, .97, .97}}
\makeatother
\makeatletter
\makeatother
\ifLuaTeX
  \usepackage{selnolig}  % disable illegal ligatures
\fi
\IfFileExists{bookmark.sty}{\usepackage{bookmark}}{\usepackage{hyperref}}
\IfFileExists{xurl.sty}{\usepackage{xurl}}{} % add URL line breaks if available
\urlstyle{same} % disable monospaced font for URLs
\hypersetup{
  pdftitle={Fundamental Epidemiology Study Designs},
  pdfauthor={Manh Nguyen Duc},
  colorlinks=true,
  linkcolor={blue},
  filecolor={Maroon},
  citecolor={Blue},
  urlcolor={Blue},
  pdfcreator={LaTeX via pandoc}}

\title{Fundamental Epidemiology Study Designs}
\author{Manh Nguyen Duc}
\date{3/20/25}

\begin{document}
\maketitle
\ifdefined\Shaded\renewenvironment{Shaded}{\begin{tcolorbox}[interior hidden, borderline west={3pt}{0pt}{shadecolor}, sharp corners, frame hidden, enhanced, boxrule=0pt, breakable]}{\end{tcolorbox}}\fi

\renewcommand*\contentsname{Table of contents}
{
\hypersetup{linkcolor=}
\setcounter{tocdepth}{2}
\tableofcontents
}
\bookmarksetup{startatroot}

\hypertarget{preface}{%
\chapter*{Preface}\label{preface}}
\addcontentsline{toc}{chapter}{Preface}

\markboth{Preface}{Preface}

I am on a journey to pursue my master's degree. I started writing this
notes as a way to summarize what I have learned from the course. It will
be basic and may contain some errors or mistakes. If you have time to go
through it, I would be grateful for your feedback. Most of the content
here is taken from the Distant Learning Epidemiology Master's Programme
provided by
\href{https://www.lshtm.ac.uk/study/courses/masters-degrees/epidemiology-online\#overview}{London
School of Hygene and Tropical Medicine}, with some modifications taken
from the book Modern Epidemiology, 4th edition (Lash et al. 2021) and
the Introduction to Biostatistics course provided by Prof.~Ronald Geskus
and his team at Oxford University Clinical Research Unit, Ho Chi Minh.

\bookmarksetup{startatroot}

\hypertarget{introduction}{%
\chapter*{Introduction}\label{introduction}}
\addcontentsline{toc}{chapter}{Introduction}

\markboth{Introduction}{Introduction}

\part{Basic concepts}

\hypertarget{fundamental-epidemiology}{%
\chapter{Fundamental epidemiology}\label{fundamental-epidemiology}}

\begin{quote}
``The study of the occurrence and distribution of health-related events,
states, and processes in specified populations, including the study of
the determinants influencing such processes, and the application of this
knowledge to control relevant health problems.'' (Porta 2014)
\end{quote}

\begin{figure}

\begin{minipage}[b]{0.50\linewidth}

{\centering 

\raisebox{-\height}{

\includegraphics{./images/John_Snow.jpg}

}

\caption{John Snow}

}

\end{minipage}%
%
\begin{minipage}[b]{0.50\linewidth}

{\centering 

\raisebox{-\height}{

\includegraphics{./images/john_snow_pump.jpg}

}

\caption{His study pump}

}

\end{minipage}%

\end{figure}

\textbf{Jhon Snow} example: For short, in 19th century, cholera expand
all over Europe and UK. Estimated 15,000 recorded deaths in London in
1848-9. Snow (an anaesthetist) came up with some hypothesis:

\begin{itemize}
\item
  that cholera can be communicated from the sick to the healthy
\item
  that disease is communicated by ``morbid matter'' which has the
  property of multiplying in the body of the person it attacks
\item
  that the morbid matter producing cholera must be introduced into the
  alimentary canal
\item
  water supplies appeared to be able to disseminate the morbid matter
  from the sick to the healthy
\end{itemize}

In 1853, cholera reappeared and Snow started his study. First he did a
descriptive study by collecting information on cholera deaths, where
they live and the population of the area. Then he look at the
association of the water sources and the risk of death from cholera.

\textbf{Comparison} is fundamental to epidemiology Just as Snow did, by
looking at the difference of risk between the groups using different
water sources, we can find the risk factors of a disease. However, in
reality, comparison is also the most challenge of epidemiology with many
sources of confusion and error.

\textbf{The two key elements} In most epidemiological studies, we would
measure 2 key elements which are the exposure and the outcome.

\begin{itemize}
\item
  The exposure (sometimes called risk factor or determinant) is any
  factor that may influence the outcome.
\item
  The outcome is the disease, or event, or health-related state, that we
  are interested in.
\end{itemize}

\hypertarget{what-is-the-role-of-epidemiology}{%
\section{what is the role of
Epidemiology?}\label{what-is-the-role-of-epidemiology}}

Epidemiology has four major functions:

\begin{itemize}
\item
  to \textbf{describe} patterns of health and disease within populations
\item
  to \textbf{interpret} these differences
\item
  to \textbf{apply} our results to public health practice, and
\item
  to \textbf{evaluate} the effect of health-related interventions
\end{itemize}

\hypertarget{basic-epidemilogical-study-types}{%
\section{Basic Epidemilogical study
types}\label{basic-epidemilogical-study-types}}

\begin{figure}

{\centering \includegraphics{./images/study_types.gif}

}

\end{figure}

This epidemiological studies family tree is simpler than the one
introduced by Ronald.

\hypertarget{measures-of-occurrence}{%
\chapter{Measures of Occurrence}\label{measures-of-occurrence}}

\hypertarget{measures-of-effects}{%
\chapter{Measures of effects}\label{measures-of-effects}}

\hypertarget{bias}{%
\chapter{Bias}\label{bias}}

\hypertarget{confounders-and-effect-modification}{%
\chapter{Confounders and Effect
Modification}\label{confounders-and-effect-modification}}

\hypertarget{measurement-error}{%
\chapter{Measurement Error}\label{measurement-error}}

\hypertarget{study-design}{%
\chapter{Study design}\label{study-design}}

\includegraphics{./images/study_types.gif}

\part{Study designs}

\hypertarget{untitled}{%
\chapter{Untitled}\label{untitled}}

\hypertarget{untitled-1}{%
\chapter{Untitled}\label{untitled-1}}

\hypertarget{untitled-2}{%
\chapter{Untitled}\label{untitled-2}}

\hypertarget{untitled-3}{%
\chapter{Untitled}\label{untitled-3}}

\bookmarksetup{startatroot}

\hypertarget{summary}{%
\chapter*{Summary}\label{summary}}
\addcontentsline{toc}{chapter}{Summary}

\markboth{Summary}{Summary}

In summary, this book has no content whatsoever.

\begin{Shaded}
\begin{Highlighting}[]
\DecValTok{1} \SpecialCharTok{+} \DecValTok{1}
\end{Highlighting}
\end{Shaded}

\begin{verbatim}
[1] 2
\end{verbatim}

\bookmarksetup{startatroot}

\hypertarget{references}{%
\chapter*{References}\label{references}}
\addcontentsline{toc}{chapter}{References}

\markboth{References}{References}

\hypertarget{refs}{}
\begin{CSLReferences}{1}{0}
\leavevmode\vadjust pre{\hypertarget{ref-lash_modern_2021}{}}%
Lash, Timothy L., Tyler J. VanderWeele, Sebastien Haneuse, and Kenneth
J. Rothman. 2021. \emph{Modern Epidemiology.} Fourth edition / Timothy
L. Lash, Tyler J. VanderWeele, Sebastien Haneuse, Kenneth J. Rothman.
Philadelphia: Wolters Kluwer.

\leavevmode\vadjust pre{\hypertarget{ref-porta_dictionary_2014}{}}%
Porta, Miquel. 2014. \emph{A Dictionary of Epidemiology.} Oxford
University Press.

\end{CSLReferences}



\end{document}
